\documentclass[12pt]{article}
%\usepackage[document]{ragged2e}
\usepackage{array, amssymb, amsthm, linguex, enumerate, amsmath, physics, enumitem, xcolor, graphicx, xparse}
\let\fg\undefined %remove linguex/siunitx naming clash
\usepackage[english]{babel}
\usepackage[letterpaper,top=2cm,bottom=2cm,left=3cm,right=3cm,marginparwidth=1.75cm]{geometry}
\usepackage[colorlinks=true, allcolors=blue]{hyperref}
\usepackage[group-separator={,}]{siunitx} %\num{12345} -> "12,345"
\usepackage{fancyhdr}
\usepackage{notomath}
\usepackage[T1]{fontenc}
\usepackage{multicol}
\usepackage{mathtools}

%Number sets
\newcommand{\R}{\mathbb{R}}
\newcommand{\C}{\mathbb{C}}
\newcommand{\N}{\mathbb{N}}
\newcommand{\F}{\mathbb{F}}
\renewcommand{\Re}{\operatorname{Re}}
\renewcommand{\Im}{\operatorname{Im}}
\renewcommand{\L}[1]{\mathcal{L}\left({#1}\right)} %Linear Map

\newcommand{\pmp}{\,\pm\,} %add small extra space to \pm

\NewDocumentCommand{\ceil}{ s m }{% ceiling brackets
    \IfBooleanTF{#1}%
    {\lceil #2 \rceil}% starred: no-autosizing
    {\left\lceil #2 \right\rceil}% unstarred: autosizing
}

\NewDocumentCommand{\ceiling}{ s m }{% ceiling brackets
    \IfBooleanTF{#1}%
    {\lceil #2 \rceil}% starred: no-autosizing
    {\left\lceil #2 \right\rceil}% unstarred: autosizing
}

\NewDocumentCommand{\floor}{ s m }{% floor brackets
    \IfBooleanTF{#1}%
    {\lfloor #2 \rfloor}% starred: no-autosizing
    {\left\lfloor #2 \right\rfloor}% unstarred: autosizing
}

\NewDocumentCommand{\pars}{ s m }{% parenthesis
    \IfBooleanTF{#1}%
    {( #2 ) }% starred: no-autosizing
    {\left( #2 \right) }% unstarred: autosizing
}

\NewDocumentCommand{\inner}{ s m }{% inner product
    \IfBooleanTF{#1}%
    {\langle #2 \rangle}% starred: no-autosizing
    {\left\langle #2 \right\rangle}% unstarred: autosizing
}

\NewDocumentCommand{\innerconj}{ s m }{% inner product
    \IfBooleanTF{#1}%
    {\overline{\langle #2 \rangle}}% starred: no-autosizing
    {\overline{\left\langle #2 \right\rangle}}% unstarred: autosizing
}

\NewDocumentCommand{\brac}{ s m }{% brackets
    \IfBooleanTF{#1}%
    {[#2] }% starred: no-autosizing
    {\left[ #2 \right] }% unstarred: autosizing
}

%default latex bracket size naming
\newcommand{\biggbrac}[1]{\bigg[ {#1} \bigg] }
\newcommand{\bigbrac}[1]{\big[ {#1} \big] }
\newcommand{\Bigbrac}[1]{\Big[ {#1} \Big] }


\RenewDocumentCommand{\over}{ s m }{% fraction 1/arg
    \IfBooleanTF{#1}%
    {\dfrac{1}{#2}}% starred: dfrac
    {\frac{1}{#2}}% unstarred: normal frac
}

\NewDocumentCommand{\pover}{ s m }{% parenthesis around fraction (1/arg)
    \IfBooleanTF{#1}%
    {\left(\dfrac{1}{#2}\right)}% starred: dfrac
    {\left(\frac{1}{#2}\right)}% unstarred: normal frac
}

\NewDocumentCommand{\pfrac}{ s m m}{% parenthesis around fraction (arg1/arg2)
    \IfBooleanTF{#1}%
    {\left( \dfrac{{#2}}{{#3}} \right)}% starred: dfrac
    {\left( \frac{{#2}}{{#3}} \right)}% unstarred: normal frac
}


\newcommand{\Xbar}{\bar{X}}
\newcommand{\Ybar}{\bar{Y}}
\newcommand{\xbar}{\bar{x}}
\newcommand{\ybar}{\bar{y}}


\newcommand{\limn}{\lim_{n\to\infty}}
\newcommand{\limx}{\lim_{x\to\infty}}

\newcommand{\gammaDist}[2]{\operatorname{Gamma} \left( {#1},{#2} \right)} %gamma distribution
\NewDocumentCommand{\normalDist}{s g g}{ %normal distibution
    \IfBooleanTF{#1} { % starred, no autosizing parenthesis
      \IfNoValueTF{#2}{
          N (\mu,\, \sigma^2 ) %\normalDist* "default" normal distribution N(\mu, \sigma^2)
        } {
            \IfNoValueTF{#3}{N (#2)}{} %\normalDist{arg} --> N(arg)
        }
      \IfNoValueTF{#3}{}{N ( #2, #3 )}  %\normalDist*{arg1}{arg2} --> N(arg1,arg2)
    }  % else (unstarred) autosize parenthesis
    {
        \IfNoValueTF{#2}{
            N \left(\mu,\, \sigma^2 \right) %\normalDist "default" normal distribution N(\mu, \sigma^2)
        } {
            \IfNoValueTF{#3}{N \left(#2\right)}{} %\normalDist{arg} --> N(arg)
        }
        \IfNoValueTF{#3}{}{N \left( #2, #3 \right)} %\normalDist{arg1}{arg2} --> N(arg1,arg2)
    }
}



%colors
\definecolor{ggreen}{RGB}{0, 127, 0}
\definecolor{dgray}{RGB}{63,63,63}
\definecolor{neonorange}{RGB}{255,47,0}
\definecolor{mygray}{rgb}{0.5,0.5,0.5}
\definecolor{eblue}{RGB}{0,74,127}
\newcommand{\red}[1]{\color{red}{#1}\color{black}}
\newcommand{\green}[1]{\color{ggreen}{#1}\color{black}}
\newcommand{\blue}[1]{\color{blue}{#1}\color{black}}
\newcommand{\setRed}{\color{red}}
\newcommand{\setBlack}{\color{black}}
\newcommand{\setBlue}{\color{blue}}
\newcommand{\setGreen}{\color{ggreen}}



\newcommand{\thru}[1]{{#1}_1, \dots, {#1}_n}
\newcommand{\sumThru}[1]{{#1}_1 + \cdots + {#1}_n}
\newcommand{\yn}{Y_1, \dots, Y_n} % Y_1, ..., Y_n
\newcommand{\xn}{X_1, \dots, X_n} % Y_1, ..., Y_n

%hats and tildes
\newcommand{\that}{\widehat{\theta}} % theta hat
\newcommand{\phat}{\widehat{p}} % p hat
\newcommand{\qhat}{\widehat{q}} % p hat
\newcommand{\psihat}{\widehat{\psi}} % psi hat
\newcommand{\Psihat}{\widehat{\Psi}} % Psi hat
\newcommand{\ptilde}{\widetilde{p}} % psi tilde
\newcommand{\Psitil}{\widetilde{\Psi}} % Psi tilde
\newcommand{\betah}{\widehat{\beta}} % beta hat
\newcommand{\bat}{\widehat{\beta}} % beta hat
\newcommand{\xhat}{\widehat{x}} % beta hat
\newcommand{\yhat}{\widehat{y}} % beta hat

%2x2 matrix shortcuts
\newcommand{\detx}[4]{\begin{vmatrix}{#1} & {#2}\\{#3}&{#4}\end{vmatrix}} % 2x2 determinant
\newcommand{\dety}[9]{\begin{vmatrix}{#1} & {#2} & {#3} \\{#4}&{#5}&{#6}\\ {#7} & {#8} & {#9}\end{vmatrix}} % 3x3 determinant
\newcommand{\bmaty}[9]{\begin{bmatrix}{#1} & {#2} & {#3} \\{#4}&{#5}&{#6}\\ {#7} & {#8} & {#9}\end{bmatrix}} % 3x3 matrix
\newcommand{\bmat}[4]{\begin{bmatrix}{#1} & {#2}\\{#3}&{#4}\end{bmatrix}} % 2x2 matrix brackets
\renewcommand{\pmat}[4]{\begin{pmatrix}{#1} & {#2}\\{#3}&{#4}\end{pmatrix}} % 2x2 matrix parenthesis

%remove any enumerate/itemize indent temporarily
\makeatletter   %% <- make @ usable in macro names
\newcommand*\notab[1]{%
  \begingroup   %% <- limit scope of the following changes
    \par        %% <- start a new paragraph
    \@totalleftmargin=0pt \linewidth=\columnwidth
    %% ^^ let other commands know that the margins have been reset
    \parshape 0
    %% ^^ reset the margins
    #1\par      %% <- insert #1 and end this paragraph
  \endgroup
}
\makeatother    %% <- revert @


\newcommand{\dimrange}[1]{\operatorname{dim}\operatorname{range}{#1}} % dimrange
\newcommand{\dimnull}[1]{\operatorname{dim}\operatorname{null}{#1}} % dimnull
\newcommand{\range}[1]{\operatorname{range}{#1}} %range
\newcommand{\nullspace}{\operatorname{null}} %null

% polynomial notation
\NewDocumentCommand{\poly}{ s g g }{%
    \IfBooleanTF{#1} {
        \IfNoValueTF{#2} {
            \mathcal{P}(\mathbb{R})
        } {
            \mathcal{P}_{#2}(\mathbb{R})
        }
    } {
        \IfNoValueTF{#3} {
            {\mathcal{P}(#2)}
        } { %else
            {\mathcal{P}_{#2}(#3)}
        }
    }
}

\NewDocumentCommand{\bias}{ s m }{% bias(arg)
    \IfBooleanTF{#1}%
    {\operatorname{bias}(#2)}% starred: no autosizing
    {\operatorname{bias}\left(#2\right)}% unstarred: autosizing
}

\NewDocumentCommand{\MSE}{ s m }{% MSE(arg)
    \IfBooleanTF{#1}%
    {\operatorname{MSE}(#2)}% starred: no autosizing
    {\operatorname{MSE}\left(#2\right)}% unstarred: autosizing
}

\NewDocumentCommand{\Var}{ s m }{% variance with parenthesis V(arg)
    \IfBooleanTF{#1}%
    {\operatorname{Var}(#2)}% starred: no autosizing
    {\operatorname{Var}\left(#2\right)}% unstarred: autosizing
}

\NewDocumentCommand{\Varb}{ s m }{% variance with brackets V[arg]
    \IfBooleanTF{#1}%
    {\operatorname{Var}[\,#2\,]}% starred: no autosizing
    {\operatorname{Var}\left[\,#2\,\right]}% unstarred: has autosizing
}

\NewDocumentCommand{\Vb}{ s m }{% another renaming of variance with brackets V[arg]
    \IfBooleanTF{#1}%
    {\operatorname{Var}[\,#2\,]}% starred: no autosizing
    {\operatorname{Var}\left[\,#2\,\right]}% unstarred: has autosizing
}

\NewDocumentCommand{\E}{ s m }{% expectation with parenthesis E(arg)
    \IfBooleanTF{#1}%
    {\operatorname{E}(#2)}% starred: no autosizing
    {\operatorname{E}\left(#2\right)}% unstarred: has autosizing
}

\NewDocumentCommand{\Eb}{ s m }{% expectation with brackets E[arg]
    \IfBooleanTF{#1}%
    {\operatorname{E}[#2]}% starred: no autosizing
    {\operatorname{E}\left[#2\right]}% unstarred: has autosizing
}

\RenewDocumentCommand{\P}{ s m }{% probability with parenthesis Pr(arg)
    \IfBooleanTF{#1}%
    {\Pr (#2) }% starred: no autosizing
    {\Pr \left( #2 \right) }% unstarred: has autosizing
}

\NewDocumentCommand{\prob}{ s m }{% probability with parenthesis Pr(arg)
    \IfBooleanTF{#1}%
    {\Pr (#2) }% starred: no autosizing
    {\Pr \left( #2 \right) }% unstarred: has autosizing
}

\NewDocumentCommand{\eff}{ s m }{% efficiency with parenthesis eff(arg)
    \IfBooleanTF{#1}%
    {\operatorname{eff}(#2)}% starred: no autosizing
    {\operatorname{eff}\left(#2\right)}% unstarred: has autosizing
}

%vertical vector of up to 8 elements
\NewDocumentCommand\vvec{s m g g g g g g g}{%
    \IfBooleanTF{#1} {
        \begin{bmatrix}% if starred use brackets
            \IfNoValueTF{#2}{}{#2}
            \IfNoValueTF{#3}{}{\\#3}
            \IfNoValueTF{#4}{}{\\#4}
            \IfNoValueTF{#5}{}{\\#5}
            \IfNoValueTF{#6}{}{\\#6}
            \IfNoValueTF{#7}{}{\\#7}
            \IfNoValueTF{#8}{}{\\#8}
        \end{bmatrix}
    }  % else (unstarred) use parethesis
    {
        \begin{pmatrix}%
            \IfNoValueTF{#2}{}{#2}
            \IfNoValueTF{#3}{}{\\#3}
            \IfNoValueTF{#4}{}{\\#4}
            \IfNoValueTF{#5}{}{\\#5}
            \IfNoValueTF{#6}{}{\\#6}
            \IfNoValueTF{#7}{}{\\#7}
            \IfNoValueTF{#8}{}{\\#8}
        \end{pmatrix}
    }
}
\def\Cov{\operatorname{Cov}} %Covariance
\def\df{\text{df}} %degrees of freedom

\NewDocumentCommand{\example}{ s g }{% Example header
    \IfBooleanTF{#1}%
    {\vspace{0.1in}}% starred: 0.1in
    {\vspace{0.2in}}% unstarred: 0.2in
    \IfNoValueTF{#2} {\noindent\textbf{\color{eblue} Example: }}{\noindent\textbf{\color{eblue} Example (#2): }}
}
\NewDocumentCommand{\disc}{ s }{% Discussion header
    \IfBooleanTF{#1}%
    {\vspace{0.1in}\noindent\textbf{Discussion: } }% starred: 0.1in
    {\vspace{0.2in}\noindent\textbf{Discussion: } }% unstarred: 0.2in
}
\NewDocumentCommand{\defn}{ s }{% Definition header
    \IfBooleanTF{#1}%
    {\vspace{0.1in}\noindent\textbf{\color{neonorange} Definition: } }% starred: 0.1in
    {\vspace{0.2in}\noindent\textbf{\color{neonorange} Definition: } }% unstarred: 0.2in
}
\NewDocumentCommand{\reason}{ s }{% Reason header
    \IfBooleanTF{#1}%
    {\vspace{0.1in}\noindent\textbf{Reason:} }% starred: 0.1in
    {\vspace{0.2in}\noindent\textbf{Reason:} }% unstarred: 0.2in
}
\NewDocumentCommand{\recall}{ s }{% Recall header
    \IfBooleanTF{#1}%
    {\vspace{0.1in}\noindent\textit{Recall:} }% starred: 0.1in
    {\vspace{0.2in}\noindent\textit{Recall:} }% unstarred: 0.2in
}
\NewDocumentCommand{\remark}{ s }{% Remark header
    \IfBooleanTF{#1}%
    {\vspace{0.1in}\noindent\textit{Remark:} }% starred: 0.1in
    {\vspace{0.2in}\noindent\textit{Remark:} }% unstarred: 0.2in
}

\NewDocumentCommand{\soln}{ s }{% Remark header
    \IfBooleanTF{#1}%
    {\vspace{0.1in}\noindent\textbf{Solution: } }% starred: 0.1in
    {\vspace{0.2in}\noindent\textbf{Solution: } }% unstarred: 0.2in
}

\newcommand{\proj}[2]{\operatorname{proj}_{{#1}}{#2}} %projection
\newcommand{\wideand}{\qquad \text{and} \qquad}

\newcommand{\bu}[1]{\textbf{\underline{{#1}}} } %bold underline
\newcommand{\boldit}[1]{\textbf{\textit{{#1}}} } %bold italix

% put actual quotation marks "around something"
\newcommand{\say}[1]{\textquotedblleft{#1}\textquotedblright}

% max{arg} and min{arg}
\renewcommand{\max}[1]{\operatorname{max}\left\{ #1 \right\}}
\renewcommand{\min}[1]{\operatorname{min}\left( #1 \right)}

\newcommand{\Span}[1]{\operatorname{span}\left\{ #1 \right\}}

%Create a new vspace line no indent
\newcommand{\nl}{\vspace{0.1in}\noindent}
\newcommand{\nnl}{\vspace{0.2in}\noindent}
\newcommand{\nnnl}{\vspace{0.3in}\noindent}
\textwidth=7.02in
\hoffset=-.425in

\setcounter{MaxMatrixCols}{20}
\begin{document}
\pagestyle{fancy}
\fancyhf{}
\fancyhead[RO]{Matthew Wilder}
\fancyhead[LO]{MTH 427 - Homework \#2}
\fancyfoot[CO]{Page \thepage}

\noindent MTH 427 - Spring 2023
\\Assignment \#1
\\Due: Monday, February 20th 2023 (11:59PM)

\section{Text Book Problems}
\begin{itemize}
\item \textbf{11.1} If $\bat_0$ and $\bat_1$ are the least-squares estimates for the intercept and slope in a simple linear regression model, show that the least-squares equation $\yhat = \bat_0 + \bat_1 x$ always goes through the point $(\xbar, \ybar)$. [\textit{Hint:} substitute $\xbar$ for $x$ in the least squares equation and use the fact that $\bat_0 = \ybar - \bat_1 \xbar$.]

\begin{proof}
Following the hint, $\yhat(\xbar) = \blue{\bat_0} + \bat_1 \xbar$. But also by the hint, $\bat_0 = \ybar - \bat_1 \xbar$, so substituting in for $\blue{\bat_0}$ gives 
$$\yhat(\xbar) = \blue{\ybar - \bat_1 \xbar} + \bat_1 \xbar = \ybar$$
Hence the point $(\xbar, \ybar)$ is a solution to the least-squares equation for any $\bat_0, \bat_1$.
\end{proof} 
\item \textbf{11.5 (use R)} What did housing prices look like in the \say{good old days}? The median sale prices for new single-family houses are given in the accompanying table for the years 1972 through 1979.${}^1$ Letting $Y$ denote the median sales price and $x$ the year (using integers $1,2, \ldots, 8$), fit the model $Y = \beta_0 + \beta_1 x + \varepsilon$. What can you conclude from the results?

  \begin{center}
    \begin{tabular}{lc}
         \hline
         Year & Median Sales Price ($\times 1000$)\\
         \hline
         1972 (1) & \$27.60 \\
         1973 (2) & \$32.50 \\
         1974 (3) & \$35.90 \\
         1975 (4) & \$39.30 \\
         1976 (5) & \$44.20 \\
         1977 (6) & \$48.80 \\
         1978 (7) & \$55.70 \\
         1979 (8) & \$62.90 \\
         \hline
    \end{tabular}
\end{center}

\soln* The summary shows that our equation is $\yhat = 4841.7x + 21575$. This means that in 1971 (year 0) the expected value of the median sales price was \$21575 and increased annually by an average of \$4841.70. 

\includegraphics*[width=4.5in]{img/11_5.PNG}
\item \textbf{11.17 (use R)} \begin{enumerate}[label=\textbf{\alph*}]
    \item Calculate $SSE$ and $S^2$ for Exercise 1.5.

    \soln* According to the R summary, the residual error is 1.746 so 
    \\$S^2 = 1746^2 = 3048516$. Then $\operatorname{SSE} = (n-2)S^2 = 6 \times 3048516 = 18291096$.

    \item It is sometimes convenient, for computational purposes, to have $x$-values spaces symmetrically and equally about zero. The $x$-values can be rescaled (or coded) in any convenient manner, with no loss of information in the statistical analysis. Refer to Exercise 1.5. Code the $x$-values (originally given on a scale of 1 to 8) by using the formula $$x^{\ast} = \frac{x-4.5}{0.5}.$$ Then fit the model $Y = \bat_0^{\ast} + \bat_1^{\ast} x^{\ast} + \varepsilon$. Calculate SSE. (Notice that the $x^{\ast}$-values are integers symmetrically spaced about zero.) Compare the SSE with the value obtained in part (a).

    \soln* Using R to compute a new summary, 

    \nl \includegraphics*[width=4.5in]{img/11_17_rb.PNG}

    Our equation is now $Y = 43362.5 + 2420.8x$.
    Then the residual error is 1.746 so 
    \\$S^2 = 1746^2 = 3048516$. Then $\operatorname{SSE} = (n-2)S^2 = 6 \times 3048516 = 18291096$. This is the same as (a), which makes sense since translating and dialating the plot along the x-values doesn't change the residual size for each data point, which determines SSE.
\end{enumerate}
\item \textbf{11.20} Suppose that $Y_1, Y_2, \ldots, Y_n$ are independent normal random variables with $\E{Y_i} = \beta_0 + \beta_1 x_i$ and $\Var{Y_i} = \sigma^2$, for $i = 1, 2, \ldots, n$. Show that the maximum-likelihood estimators (MLEs) of $\beta_0$ and $\beta_1$ are the same as the least-squares estimators of section 11.3

\soln*
$$f(y_i) = \over{\sqrt{2\pi} \sigma} \cdot \exp{-\over{2\sigma^2} (y_i - \beta_0 - \beta_1 x_i)^2}$$
\begin{align*}
    L(Y_1, \ldots, Y_n \mid \beta_0, \beta_1) &= \prod_{i=1}^n \over{\sqrt{2\pi} \sigma} \cdot \exp{-\over{2\sigma^2} (y_i - \beta_0 - \beta_1 x_i)^2}
    \\ &= \pover{\sqrt{2\pi} \sigma}^n \prod_{i=1}^n  \cdot \exp{-\over{2\sigma^2} (y_i - \beta_0 - \beta_1 x_i)^2}
    \\ &= \pover{\sqrt{2\pi} \sigma}^n \cdot \exp{-\over{2\sigma^2} \sum_{i=1}^n  (y_i - \beta_0 - \beta_1 x_i)^2}\\
\end{align*}
\begin{align*}
l(Y_1, \ldots, Y_n \mid \beta_0, \beta_1) &= \ln \brac{\pover{\sqrt{2\pi} \sigma}^n \cdot \exp{-\over{2\sigma^2} \sum_{i=1}^n  (y_i - \beta_0 - \beta_1 x_i)^2}} \\
    &= \ln \pars{\pover{\sqrt{2\pi} \sigma}^n} + \ln \pars{\exp{-\over{2\sigma^2} \sum_{i=1}^n  (y_i - \beta_0 - \beta_1 x_i)^2}} \\
    &= n \ln \pover{\sqrt{2\pi} \sigma}  -\over{2\sigma^2} \sum_{i=1}^n  (y_i - \beta_0 - \beta_1 x_i)^2
\end{align*}
Then $$\pdv{l}{\beta_0} = -\over{\sigma^2} \sum_{i=1}^n  (y_i - \beta_0 - \beta_1 x_i) = 0 \implies -n\beta_0 + \sum_{i=1}^n y_i - \beta_1 \sum_{i=1}^n x_i = 0$$
So $\sum_{i=1}^n y_i = n\beta_0 + \beta_1 \sum_{i=1}^n x_i$. Solving for $\beta_0$,
\begin{align*}
    \sum_{i=1}^n y_i &= n\beta_0 + \beta_1 \sum_{i=1}^n x_i \\
    n\beta_0 &= \sum_{i=1}^n y_i - \beta_1 \sum_{i=1}^n x_i\\
    \beta_0 &= \ybar - \beta_1 \xbar
\end{align*}

\nnl Next $$\pdv{l}{\beta_1} = -\over{\sigma^2} \sum_{i=1}^n  \pars{x_i(y_i - \beta_0 - \beta_1 x_i)} = 0$$
So $\sum_{i=1}^n x_i y_i = \beta_0 \sum_{i=1}^n x_i + \beta_1 \sum_{i=1}^n x_i^2$

Now solving for $\beta_1$, 
\begin{align*}
    \sum x_i y_i &= \beta_0 \sum x_i + \beta_1 \sum x_i^2\\
    \beta_1 &= \frac{\sum x_i y_i - \beta_0 \sum x_i }{ \sum x_i^2} \\
    &= \frac{\sum x_i y_i -  \blue{\pars{ \ybar - \beta_1 \xbar}} \sum x_i }{ \sum x_i^2}\\
    &=  \frac{\sum x_i y_i -  \ybar \sum x_i + \beta_1 \xbar \sum x_i }{ \sum x_i^2}\\
    \beta_1 - \frac{\beta_1 \xbar \sum x_i}{ \sum x_i^2} &= \frac{\sum x_i y_i -  \ybar \sum x_i}{ \sum x_i^2}\\
    \beta_1 \pars{ \frac{\sum x_i^2 - \xbar \sum x_i}{\sum x_i^2} } &= \frac{\sum x_i y_i -  \ybar \sum x_i}{ \sum x_i^2}\\
    \beta_1 &= \frac{\sum x_i y_i -  \ybar \sum x_i}{ \sum x_i^2 - \xbar \sum x_i}\\
    &= \frac{\sum x_i y_i - \over{n} \sum y_i \sum x_i}{ \sum x_i^2 - \over{n} (\sum x_i)^2}\\
    &= \frac{S_{xy}}{S_{xx}}
\end{align*}

\nnl Which is the same as the least-squares method achieved.
\item \textbf{11.21} Under the assumptions of Exercise 11.20, find $\Cov(\bat_0, \bat_1)$. Use this answer to show that $\bat_0$ and $\bat_1$ are independent if $\sum_{i=1}^n x_i = 0$. [\textit{Hint:} $\Cov(\bat_0, \bat_1) = \Cov(\Ybar - \bat_1 \xbar, \, \bat_1)$. Use Theorem 5.12 and the results of this section.]

\soln* $ $
\begin{align*}
    \Cov(\bat_0, \bat_1) &= \Cov(\Ybar - \bat_1 \xbar, \bat_1) \tag{using the hint}\\
    &= \Cov(\Ybar, \bat_1) + \Cov(- \bat_1 \xbar, \bat_1) \tag{separate sum}\\
    &= \underbrace{\Cov(\Ybar, \bat_1)}_{0} + \Cov(- \bat_1 \xbar, \bat_1) \tag{by Theorem 5.12 and page 579}\\
    &= \Cov(- \bat_1 \xbar,\; \bat_1) \\
    &= \Cov \pars{-\bat_1 \cdot \over{n} \sum_{i=1}^n x_i,\; \bat_1} \tag{definition of $\xbar$}\\
    &= \Cov \pars{-\bat_1 \cdot \over{n} \cdot 0,\; \bat_1} \tag{by hypothesis}\\
    &= \Cov(0, \bat_1)\\
    &= 0 \tag{by Covariance of a constant}
\end{align*}
Therefore $\bat_0$ and $\bat_1$ are independent since their Covariance equals zero.
\item \textbf{11.30a} In both cases, $H_0 : \bat_1 = 0$ vs $H_a : \bat_1 \neq 0$ with $\alpha = 0.05$.
\begin{enumerate}[label=]
    \item \textbf{Small} $$S^2 = \frac{\operatorname{SSE}}{n-2} = \frac{2.04}{29} = 0.070345 \implies S = 0.26523$$
    $$\Var*{\bat_1} = c_{1\,1}S^2 = (0.0202)^2 = c_{1\,1} \cdot 0.070345 \implies c_{1\,1} = 0.0058$$
    Computing the test statistic, 
    $$T = \frac{\bat_1 - 0}{S\sqrt{c_{1\,1}}} = \frac{0.155}{0.26523 \sqrt{0.0058}} = 7.674$$
    Using the stat tables, $t_{0.025,\,29} = 2.045$, hence we are in the rejection region and the slope is nonzero.
    \item \textbf{Large} $$S^2 = \frac{\operatorname{SSE}}{n-2} = \frac{1.86}{9} = 0.20667 \implies S = 0.4546$$
    $$\Var*{\bat_1} = c_{1\,1}S^2 = (0.0193)^2 = c_{1\,1} \cdot 0.20667 \implies c_{1\,1} = 0.0018$$
    Computing the test statistic, 
    $$T = \frac{\bat_1 - 0}{S\sqrt{c_{1\,1}}} = \frac{0.190}{ 0.4546 \sqrt{0.0018}} = 9.851$$
    Using the stat tables, $t_{0.025,\,9} = 2.262$, hence we are in the rejection region and the slope is nonzero.
\end{enumerate}
Thus both slopes are significantly far different than 0.
\end{itemize}
\section{Additional Exercises Using R}
\subsection{Exercise 1}
\textit{This exercise relates to the \say{Hwk-data2} dataset}
One study enrolled a group of 10 nurses, ages 50-54 years, who had smoked at least 1
pack per day and quit for at least 6 years. The nurses reported their weight before and
6 years after quitting smoking. A commonly used measure of obesity is BMI = $w/h^2$
$(\text{weight}/\text{height}^2)$. The BMI of the 10 women before and 6 years after quitting smoking are
given in the last two columns of: \say{Hwk-data2.csv}

\begin{enumerate}[label=(\alph*)]
    \item What test can be used to asses whether the mean BMI changed among heavy-smoking
women 6 years after quitting smoking? Specify the hypotheses.

\soln* A paired t-test can be used with $H_0 : \mu_d = 0$ vs $H_a : \mu_d \neq 0$.
    \item Implement the test in part(a). (Is there sufficient evidence that the mean BMI changed
among heavy-smoking women 6 years after quitting smoking?)

\soln* Yes, there is sufficient evidence that the BMI changed after 6 years (since the p-value is very small).

\nl \includegraphics*[width=5in]{img/2_1a_code.PNG}

\nl \includegraphics*[width=4in]{img/2_1a_console.PNG}
    \item Provide a 98\% confidence interval for the true mean change in BMI among heavysmoking women.

\soln* The data is normal, so the 98\% confidence interval is $(1.162738, 5.557262)$.

\nl \includegraphics*[width=5in]{img/2_1c_code.PNG}

\nl \includegraphics*[width=4in]{img/2_1c_console.PNG}

\notab{One issue is that there has been a secular change in weight in society. For this purpose,
a control group of 50-to 54 year old never-smoking women were recruited and their BMI was
reported at baseline (ages 50-54) and 6 years later at a follow-up visit. The results are given
in the first two columns of: \say{Hwk-data2.csv}}
    \item What test can be used to assess whether the mean change in BMI over 6 years is
different between women who quit smoking and women who have never smoked? Specify
the hypotheses.

\soln* A two sample t-test, pooled (as the following R code shows the variances are equal, since the p-value is large). $H_0 : \mu_1 = \mu_2$ vs $H_a : \mu_1 \neq \mu_2$, where $\mu_1$ is the mean difference of the baseline group, and $\mu_2$ is the mean difference of the smokers.

\nl \includegraphics*[width=4in]{img/2_1d_console.PNG}
    \item Implement the test in part (d) (Do the data provide sufficient evidence to indicate a
difference in mean BMI between the heavy-smoking women 6 years after quitting smoking
and the never-smoking women at 6-year follow-up.)

\soln* Since the p value is large ($> 0.1$ in this case), we fail to reject the null hypothesis, and hence there is insufficient evidence that the smokers and non smokers BMI is difference. 

\nl \includegraphics*[width=4in]{img/2_1e_console.PNG}
    \item Provide a 90\% Confidence interval for the difference in mean BMI between the heavysmoking women
6 years after quitting smoking and the never-smoking women at 6-year follow-up.

\soln* A 90\% confidence interval is $(-0.03178454, 3.65178454)$.

\nl \includegraphics*[width=5in]{img/2_1f_console.PNG}
\end{enumerate}
\subsection{Exercise 2}
\textit{This exercise relates to the \say{Auto} dataset}
\begin{enumerate}[label=(\alph*)]
    \item Use the appropriate function in R to perform a simple linear regression with \textit{mpg} as
the response variable and \textit{horsepower} as the predictor.

\soln*  $ $

\nl \includegraphics*[width=5in]{img/2_2a_console.PNG}
    \item Give an interpretation of the coefficients in term of \textit{mpg} and \textit{horsepower}

\soln* The intercept $\beta_0$ means that a car with zero horsepower is expected to get 39.9 mpg. The slope coefficient $\beta_1$ means that for every unit increase of horsepower, the car's mpg will drop an average of 0.1578 units.
    \item Test whether there is a linear relationship between the predictor and the response?
(i.e test whether the regression coefficient (slope) is zero: $H_0 : \beta_1 = 0$ vs $H_a : \beta_1 \neq 0$)

\soln* Since the R summary says $p < 2 \times 10^{-16}$, we are extremely confident the slope is non-zero.
    \item Use the appropriate function in R to obtain 98\% confidence intervals of the coefficient(s).

\soln* The intercept $\beta_0 \in (38.2598220, 41.6119001)$ and slope $\beta_1 \in (-0.1729011, -0.1427884)$

\nl \includegraphics*[width=5in]{img/2_2d_console.PNG}
    \item Display a scatter plot between \textbf{mpg} and \textbf{horsepower}. Does the scatter plot suggest
a linear relationship between the two variables? Explain why?

\soln* The plot suggests some linearity since the data is clustered in a downwards trend, but it looks closer to a graph of $y(x)=1/x$. The correlation coefficient from the summary is $0.6049$, so it has a regular amount of correlation (not strong, not weak).

\nl \includegraphics*[width=1.5in]{img/2_2e_code.PNG}

\nl \includegraphics*[width=5in]{img/2_2e_plot.PNG}
    \item Display the least square regression line in the scatter plot in (a).

\soln* $ $

\nl \includegraphics*[width=3.5in]{img/2_2f_code.PNG}

\nl \includegraphics*[width=5in]{img/2_2f_plot.PNG}
\end{enumerate}

\end{document}
